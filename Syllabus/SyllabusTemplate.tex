\documentclass[11pt]{article}
\usepackage{amssymb}
\pagestyle{empty}
\usepackage{color}
\usepackage[colorlinks=true,urlcolor=blue]{hyperref}
\usepackage[T1]{fontenc}
\usepackage{lmodern}
%\usepackage{palatino}
\newcommand{\alert}[1]{\textcolor{red}{#1}}
\def\UrlFont{\tt}


\begin{document}
\begin{center}



{
\LARGE { PSTAT 126: Regression Analysis}\\
\normalsize {Summer 2023 }
}

\vspace{.3in}
{\large{\bf Course Policies and Syllabus}}
\vspace{.25in}
\end{center}

\vspace{0.1in}

\noindent{\bf Course website:}  {\fontfamily{cmtt}\selectfont \url{https://ucsb.instructure.com/courses/13351}} 

\vspace{0.1in}

\begin{tabular}{l} 
{\fontfamily{lmr}\selectfont\bf Instructor:} Laura Baracaldo     \\
{\bf Email:}  {\fontfamily{cmtt}\selectfont lnbaracaldol@ucsb.edu} \\
 {\bf Office Hours:} Wed 1:45-2:45pm. Old Gym 1201. \\
 %\quad Zoom link: {\fontfamily{cmtt}\selectfont \url{https://ucsb.zoom.us/j/82985864110?pwd=ODZFMFhmZm0xbHZxKzRheStyZ09WUT09}} \\
%\quad Meeting ID: 829 8586 4110\\
%\quad Passcode: 091614\\
\\
{\fontfamily{lmr}\selectfont\bf TA:} Moya Xiong      \\
{\bf Email:}  {\fontfamily{cmtt}\selectfont 
moyaxiong@pstat.ucsb.edu}  \\
 {\bf Office Hours:} TBD\\
 \\ 


\end{tabular}


\vspace{.2in}

\noindent
\qquad {\fontfamily{lmr}\selectfont\bf Lecture:}   \quad  	MTWR 12:30- 1:35 BRDA 1640    

\vspace{.15in}

\noindent% TODO
\qquad {\bf Sections:}
\begin{tabular}{l c l l c}
 {\bf Section \#}   &   {\bf Days}   & {\bf Time}   & {\bf Location} & {\bf TA}  \\ 
     15339   &M W& 	 9:00- 9:50 am   & PHELP 1513 & 	Moya Xiong \\  
     15347  & MW&   10:00-10:50 am	   & PHELP 1513  & 	Moya Xiong  \\   
 
\end{tabular}


\vspace{0.2in}

\noindent{\bf Course Topics and Objectives:}

\vspace{0.1in}

\noindent This course introduces the theory and application of linear regression models. We will give a
(non-exhaustive) introduction to linear models in an increasing order of complexity throughout
this quarter. Topics may include simple and multiple regression models; estimation; inference;
prediction; regression diagnostics; model selection; shrinkage methods; analysis of variance, and
more if time allows. The course will be focused on the learning of computer package R to solve real-world problems.

\vspace{0.1in}

Upon completion of this course, a student should be able to:

\vspace{0.1in}

\begin{itemize}
\item Explain the common regression techniques conceptually and characterize some of them
mathematically;
\item Describe and discuss the concepts, use cases, and properties of the linear model and many
of its extensions;
\item Be able to effectively use R for exploratory data analysis, model fitting, and visualization.
\end{itemize}


\vspace{0.1in}

\noindent{\bf Course Material:}

\begin{itemize}
\item Lecture slides and labs will be available on {\em Canvas}. All the course material will be weekly posted in the {\em Modules} section as the quarter progresses. 
\item Homework assignments and quizzes will also be given out on {\em Canvas}, and should be turned in on
{\em Gradecope}. Other forms of submission will not be accepted.
\item All Q\&A related to course content, homework assignments, R programming, and quizzes should
be done on Nectir. 
\end{itemize}

\vspace{0.1in}



\noindent{\bf Prerequisites:} PSTAT 10 and PSTAT 120B both with a minimum grade of C or better. This course
heavily depends on matrix algebra (MATH 4A). Some familiarity with R is expected.

\vspace{0.1in}

\noindent{\bf References:} The lecture slides are self-contained. You may find the following textbooks helpful:

\begin{itemize}
\item Faraway, J. J. (2005),\textit{ Linear Models with R}, Chapman \& Hall.
\item Weisberg, S. (2005),\textit{ Applied Linear Regression}, 3rd edition, Wiley.
\end{itemize}

\noindent{\bf R programming} 

\vspace{0.1in}

\noindent We will heavily use the {\fontfamily{cmtt}\selectfont R}  programming language ( {\fontfamily{cmtt}\selectfont \url{www.r-project.org}} ) throughout this course.
Please set up your {\fontfamily{cmtt}\selectfont R}  environment, if you have not, as early as possible.

\vspace{0.1in}

\noindent Reading the following book (which is freely available online) is very helpful for quickly coming
up to speed with {\fontfamily{cmtt}\selectfont R}.

\begin{itemize}
\item \textit{R for Data Science} by Grolemund and Wickham (available at:  \url{ https://r4ds.had.co.nz/index.html})
\end{itemize}


\vspace{0.1in}

\noindent{\bf Course Grading:}

\begin{itemize} 
\item {\bf Homework} (40\%). 
\begin{itemize}
\item There will be 4 homework assignments, due approximately every 2 weeks. Each assignment
will be worth the same amount.
\item {\bf Homework solutions must be done using RMarkdown and turned in on Gradescope}.
All code should be well documented. We have provided a homework
template that you can use to get started. {\bf You MUST submit the R Markdown code ({\fontfamily{cmtt}\selectfont Rmd} file, template provided), the PDF generated and any additional file as needed}.
\item Homework not submitted online before the deadline will be considered late (20\%
deduction from the received assignment credit). 24 hours after the deadline homework
will not be accepted and no credit will be awarded. Since internet service can be
unreliable, you are encouraged to start and submit your assignments well in advance
of the deadline.
\item { \bf You may not copy or make use of solutions from the web, other students, or other
sources.}
\end{itemize}
\item {\bf Quizzes} (20\%).
\begin{itemize}
\item There will be occasional quizzes given on Canvas. Each quiz will be worth the
same amount.
\item There is no make-up for missed quizzes.
\end{itemize}
\item {\bf Final exam}(40\%). 

\end{itemize}

\vspace{0.2in}



\noindent {\bf Code of Academic Integrity}

\vspace{0.1in}

\noindent It is expected that you will adhere to the UCSB \textit{Student Conduct Code} at {\fontfamily{cmtt}\selectfont\url{http://studentconduct.sa.ucsb.edu/academic-integrity}}. In quizzes, homework assignments, and final project, you may not copy or make use of solutions from the web, other (groups of) students, or other sources.

\vspace{0.1in}

\noindent The course materials, including lecture slides, lecture recordings, quizzes, homework assignments,
and similar materials, are protected by U.S. copyright law and by University policy. You
may take notes and make copies of course materials for your own use. You may also share those
materials with another student who is enrolled in or auditing this course. Do not reproduce,
distribute or display (post/upload) course materials in any other way -- whether or not a fee is
charged -- without the express prior written consent from the lecturer. To do so will result in a
UCSB Honor Code investigation.

\vspace{0.2in}

\noindent{\bf Grade Appeals}

\vspace{0.1in}

\noindent Grade appeals must be made to your TA, in writing, no sooner than 24 hours after the assignment or exam is returned, and no later than 4 days after it is returned. Please provide written
justification for your appeal and include the homework or exam in question, along with any
relevant supplementary information.

\vspace{0.1in}

\noindent If you have a dispute with your TA over a grade you have received, you have the right to request
a review by the professor. Please keep in mind, however, that an appeal will invoke a review of
the full assignment and could result in an even lower grade.

\vspace{0.2in}

\noindent{\large \bf COVID-19 Health and Safety Requirement}

\vspace{0.1in}

\noindent Follow the policies and requirements listed at \url{https://www.ucsb.edu/COVID-19-information}.
Non-compliance with COVID-19 health and safety requirements is a violation of the UCSB Student
Code of Conduct.

\vspace{0.2in}

\noindent{\large \bf Links for Campus Resources}



\begin{itemize}
\item Office of Ombuds \url{https://ombuds.ucsb.edu} provides confidential consultation services to
faculty, staff, students, parents, or anyone else with a campus-related concern.
\item DSP. \url{https://dsp.ext-prod.sa.ucsb.edu}. Students that will need special assistance must contact me as soon as possible and have their Disabled Student Program Specialist send me a formal request.
\end{itemize}






%\noindent \underline{Please attend the first lab for orientation.} Subsequent lab attendance is NOT
%required.
\vspace{0.3in}







\end{document}

\noindent {\bf Lab Assignments:} Lab assignments will be completed,
submitted, and reviewed in {\em Canvas}.  All lab assigments will be posted in the
{\em Modules} section. Assignments will be comprised of {\em Canvas Quizzes} and {\em iMathAS} 
question sets. Data files used in labs can be accessed in the {\em Modules} themselves, 
but they are also available for download in the {\em Files} tab on {\em Canvas}.



\noindent {\bf Late Work:} Late submissions will {\bf NOT} be accepted. The
class accommodates missing Lab assignments by designating Labs 5 and 10 as
extra credit (see {\bf Course Grade} section below). Therefore, instructors
will adhere to a strict assignment submission policy. Complete the labs early
in the week. Do not wait until the day the assignments are due! In cases of
extenuating circumstances (typically requiring documentation, i.e. doctor's note), accommodating late work will be left at the
discretion of the instructors. In such cases, email both instructors at least
48 hours before the due date of the assignment.
\vspace{0.2in}
 
 

\newpage
\noindent {\bf Schedule and Content List:}  \vspace{.05in}  \\ 
\indent \begin{tabular}{| l | l |p{10cm}|} \hline
Lab \# & Due Date & Content \\ \hline
Lab 1 & 10/07, 9 am &Practice with Data Types, Starting JMP. \\ \hline
Lab 2 & 10/14, 9 am & Looking at data. Measures of central tendency, Measures of dispersion. \\ \hline

Lab 3 & 10/21, 9 am &Relative Frequency, Probability (including Bayes Theorem), Binomial and Poisson distribution.\\ \hline

Lab 4 & 10/28, 9 am & Means of Normals, Central Limit Theorem, Normal Approximation to Binomial\\ \hline

Lab 5 & 11/04, 9 am & EXTRA CREDIT. Review lab.\\ \hline

Lab 6 & 11/11, 9 am &Confidence Intervals for Means, Confidence Intervals for Proportions.\\ \hline

Lab 7 & 11/18, 9 am &One Sample Hypothesis Tests for Means, Hypothesis Tests for Proportions. Two-sample Tests for Means.\\ \hline

Lab 8&  11/25, 9 am &Regression, Residuals and Transformations\\ \hline

Lab 9& 12/02, 9 am & Multiple Regression, Goodness-of-Fit Tests\\ \hline

Lab 10 & \textbf{12/09, 9 am} & EXTRA CREDIT. Polynomial Regression, Optimization.\\ \hline
\end{tabular}

\vspace{.3in}

\noindent {\bf Course Grade:} Grades will be based on a point system. Each
question within a required lab is worth one point. The total number of
questions in the eight required labs is approximately 250, and the primary
grade percentage will be calculated out of the total. Extra credit labs carry
an additional 20\% total, more than enough to replace an entire missed lab.
The final score (the raw percentage plus extra credit) will determine a
student's letter grade: 90\% - 100\% is an A, 80\% - 89\% is a B, 70\% - 79\%
is a C, 60\% - 69\% is a D, and 0 - 59\% is an F. Note that A+ will not be
given for students who finish higher than 100\%. We will not round or bargain
for scores that are borderline between different grade levels. \\

\noindent {\bf DRC Accommodations:} The Disability Resources Center reduces barriers to inclusion and full participation for students
with disabilities by providing support to individually determine reasonable academic
accommodations. If you have questions or concerns about accommodations or any other
disability-related matter, please contact the DRC office, located in Hahn 125 or at 831-459-2089
or \href{mailto:drc@ucsc.edu}{drc@ucsc.edu}. \\

\noindent {\bf Academic Dishonesty:} Academic integrity is the cornerstone of a university education. Academic dishonesty diminishes
the university as an institution and all members of the university community. It tarnishes the
value of a UCSC degree.

All members of the UCSC community have an explicit responsibility to foster an environment of
trust, honesty, fairness, respect, and responsibility. All members of the university community are
expected to present as their original work only that which is truly their own. All members of the
community are expected to report observed instances of cheating, plagiarism, and other forms
of academic dishonesty in order to ensure that the integrity of scholarship is valued and
preserved at UCSC.

In the event a student is found in violation of the UCSC Academic Integrity policy, he or she may
face both academic sanctions imposed by the instructor of record and disciplinary sanctions
imposed either by the provost of his or her college or the Academic Tribunal convened to hear
the case. Violations of the Academic Integrity policy can result in dismissal from the university
and a permanent notation on a student�s transcript.

For the full policy and disciplinary procedures on academic dishonesty, students and instructors
should refer to the \href{https://www.ue.ucsc.edu/academic_misconduct}{Academic Integrity page} at the Division of Undergraduate Education. \\

\noindent {\bf Title IX:} The university cherishes the free and open exchange of ideas and enlargement of knowledge.
To maintain this freedom and openness requires objectivity, mutual trust, and confidence; it
requires the absence of coercion, intimidation, or exploitation. The principal responsibility for
maintaining these conditions must rest upon those members of the university community who
exercise most authority and leadership: faculty, managers, and supervisors.
The university has therefore instituted a number of measures designed to protect its community
from sex discrimination, sexual harassment, sexual violence, and other related prohibited
conduct. \href{https://titleix.ucsc.edu/}{Information about the Title IX Office}, the \href{https://ucsc-gme-advocate.symplicity.com/public_report/index.php/pid681212?}{online reporting link}, applicable campus
\href{https://titleix.ucsc.edu/Resources\%20and\%20Options\%202.18.pdf}{resources}, reporting responsibilities, the \href{https://policy.ucop.edu/doc/4000385/SVSH}{UC Policy on Sexual Violence and Sexual Harassment}
and the UC Santa Cruz Procedures for Reporting and Responding to Reports of Sexual
Violence and Sexual Harassment can be found at \href{titleix.ucsc.edu}{titleix.ucsc.edu}.
The Title IX/Sexual Harassment Office is located at 105 Kerr Hall. In addition to the \href{https://ucsc-gme-advocate.symplicity.com/public_report/index.php/pid681212?}{online
reporting option}, you can contact the Title IX Office by calling 831-459-2462. 
